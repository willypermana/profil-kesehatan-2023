\phantomsection
\addcontentsline{lot}{section}{\protect\numberline{}Tabel 45 - Cakupan pemberian vitamin A pada bayi dan anak balita}
\ra{1.3}
%Table generated by Excel2LaTeX from sheet '45'

{\centering
\begin{tabular}{rY{9em}Y{8em}Z{4em}Z{4em}Z{4em}Z{4em}Z{4em}Z{4em}Z{4em}Z{4em}Z{4em}}
    \multicolumn{12}{l}{Tabel 45}\\
    \multicolumn{12}{c}{CAKUPAN PEMBERIAN VITAMIN A PADA BAYI DAN ANAK BALITA MENURUT KECAMATAN DAN PUSKESMAS}\\
    \multicolumn{12}{c}{KABUPATEN BELITUNG TIMUR}\\
    \multicolumn{12}{c}{TAHUN \tP}\\
    \toprule
    \multirow{3}[0]{*}{NO} & \multirow{3}[0]{*}{KECAMATAN} & \multirow{3}[0]{*}{PUSKESMAS} & \multicolumn{3}{c}{BAYI 6-11 BULAN\textsuperscript{1}} & \multicolumn{3}{c}{ANAK BALITA (12-59 BULAN)} & \multicolumn{3}{c}{BALITA (6-59 BULAN)} \\
    \cmidrule(l{2pt}r{2pt}){4-6} \cmidrule(l{2pt}r{2pt}){7-9} \cmidrule(l{2pt}r{2pt}){10-12}
    & & & \multirow{3}[0]{4em}{\raggedleft JUMLAH BAYI} & \multicolumn{2}{c}{MENDAPAT VIT A} & \multirow{3}[0]{4em}{JUMLAH } & \multicolumn{2}{c}{MENDAPAT VIT A} & \multirow{3}[0]{4em}{JUMLAH} & \multicolumn{2}{c}{MENDAPAT VIT A} \\
    \cmidrule{5-6}\cmidrule{8-9}\cmidrule{11-12}
%    & & & & \multicolumn{1}{c}{$\Sigma$} & \multicolumn{1}{c}{\%} & & \multicolumn{1}{c}{$\Sigma$} & \multicolumn{1}{c}{\%} & & \multicolumn{1}{c}{$\Sigma$} & \multicolumn{1}{c}{\%} \\
    & & & & Jml & \% & & Jml & \% & & Jml & \% \\
    \midrule
    \emph{1} & \emph{2} & \emph{3} & \emph{4} & \emph{5} & \emph{6} & \emph{7} & \emph{8} & \emph{9} & \emph{10} & \emph{11} & \emph{12} \\
    \midrule
	1 & Manggar           & Manggar       &   642 &   628 &  97,89 & 2.017 & 1.931 &  95,74 & 2.659 & 2.559 &  96,26 \\
	2 & Damar             & Mengkubang    &   214 &   203 &  94,92 &   750 &   750 & 100,00 &   964 &   953 &  98,87 \\
	3 & Kelapa Kampit     & Kelapa Kampit &   310 &   249 &  80,43 &   995 &   995 & 100,00 & 1.305 & 1.244 &  95,36 \\
	4 & Gantung           & Gantung       &   465 &   815 & 175,19 & 1.370 & 1.208 &  88,18 & 1.835 & 2.023 & 110,23 \\
	5 & Simpang Renggiang & Renggiang     &   122 &   127 & 104,40 &   424 &   418 &  98,58 &   546 &   545 &  99,88 \\
	6 & Simpang Pesak     & Simpang Pesak &   138 &   103 &  74,53 &   426 &   426 & 100,00 &   564 &   529 &  93,76 \\
	7 & Dendang           & Dendang       &   172 &   154 &  89,28 &   650 &   637 &  98,00 &   822 &   791 &  96,17 \\
    \midrule
    \multicolumn{3}{l}{JUMLAH}            & 2.063 & 2.279 & 110,50 & 6.632 & 6.365 &  95,97 & 8.695 & 8.644 &  99,42 \\
    \bottomrule
\end{tabular}%

}

\vspace{2ex}
{\small
	\textsuperscript{1} Keterangan: Pelaporan pemberian vitamin A dilakukan pada Februari dan Agustus, maka perhitungan bayi 6-11 bulan yang mendapat vitamin A dalam setahun dihitung dengan mengakumulasi bayi 6-11 bulan yang mendapat vitamin A di bulan Februari dan yang mendapat vitamin A di bulan Agustus. Sehingga jumlah sasaran bayi 6-11 bulan = jumlah bayi setahun.
	
}
\vfill
Sumber: Subkoordinator Kesehatan Keluarga dan Gizi\par 
