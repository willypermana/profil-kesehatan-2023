\phantomsection
\addcontentsline{lot}{section}{\protect\numberline{}Tabel 60 - Jumlah ODHIV baru yang mendapat pengobatan}
\ra{1.3}

%Table generated by Excel2LaTeX from sheet '60'
{\centering
\begin{tabular}{rY{9em}Y{8em}Z{8em}Z{8em}Z{8em}}
    \multicolumn{6}{l}{Tabel 60}\\
    \multicolumn{6}{c}{PRESENTASE ODHIV BARU MENDAPATKAN PENGOBATAN MENURUT KECAMATAN DAN PUSKESMAS}\\
    \multicolumn{6}{c}{KABUPATEN BELITUNG TIMUR}\\
    \multicolumn{6}{c}{TAHUN \tP}\\
    \toprule
    NO & KECAMATAN & PUSKESMAS & ODHIV BARU DITEMUKAN & ODHIV BARU DITEMUKAN DAN MENDAPAT PENGOBATAN ARV & PERSENTASE ODHIV BARU MENDAPAT PENGOBATAN ARV \\
    \midrule
    \emph{1} & \emph{2} & \emph{3} & \emph{4} & \emph{5} & \emph{6} \\
    \midrule
	1 & Manggar           & Manggar       & 13 &  8 &  61,54 \\
	2 & Damar             & Mengkubang    &  2 &  1 &  50,00 \\
	3 & Kelapa Kampit     & Kelapa Kampit &  4 &  4 & 100,00 \\
	4 & Gantung           & Gantung       &  3 &  3 & 100,00 \\
	5 & Simpang Renggiang & Renggiang     &  0 &  0 &   0,00 \\
	6 & Simpang Pesak     & Simpang Pesak &  2 &  2 & 100,00 \\
	7 & Dendang           & Dendang       &  2 &  1 &  50,00 \\
    \midrule
    \multicolumn{3}{l}{JUMLAH KAB.}       & 26 & 19 &  73,08 \\
    \bottomrule
\end{tabular}%

}

\vfill
Sumber: Subkoordinator Pengendalian Penyakit Menular\par 