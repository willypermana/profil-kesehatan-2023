\phantomsection
\addcontentsline{lot}{section}{\protect\numberline{}Tabel 64 - Kasus baru kusta menurut jenis kelamin}
\ra{1.3}
%Table generated by Excel2LaTeX from sheet '64'

{\centering
\begin{tabular}{rY{9em}Y{8em}Z{4em}Z{4em}Z{4em}Z{4em}Z{4em}Z{4em}Z{4em}Z{4em}Z{4em}}
    \multicolumn{12}{l}{Tabel 64}\\
    \multicolumn{12}{c}{KASUS BARU KUSTA MENURUT JENIS KELAMIN, KECAMATAN, DAN PUSKESMAS}\\
    \multicolumn{12}{c}{KABUPATEN BELITUNG TIMUR}\\
    \multicolumn{12}{c}{TAHUN \tP}\\
    \toprule
    \multirow{3}[0]{*}{NO} & \multirow{3}[0]{*}{KECAMATAN} & \multirow{3}[0]{*}{PUSKESMAS} & \multicolumn{9}{c}{KASUS BARU} \\
    \cmidrule{4-12}
    & & & \multicolumn{3}{X{15em}}{Pausi Basiler (PB)/ Kusta kering} & \multicolumn{3}{X{15em}}{Multi Basiler (MB)/ Kusta Basah} & \multicolumn{3}{X{15em}}{PB + MB} \\
    \cmidrule(l{2pt}r{2pt}){4-6}\cmidrule(l{2pt}r{2pt}){7-9}\cmidrule(l{2pt}r{2pt}){10-12}
    & & & L & P & L+P & L & P & L+P & L & P & L+P \\
    \midrule
    \emph{1} & \emph{2} & \emph{3} & \emph{4} & \emph{5} & \emph{6} & \emph{7} & \emph{8} & \emph{9} & \emph{10} & \emph{11} & \emph{12} \\
    \midrule
	1 & Manggar           & Manggar       &         0 &    0 & 0 &     1 &     1 & 2 &     1 &     1 & 2 \\
	2 & Damar             & Mengkubang    &         0 &    0 & 0 &     0 &     0 & 0 &     0 &     0 & 0 \\
	3 & Kelapa Kampit     & Kelapa Kampit &         0 &    0 & 0 &     0 &     0 & 0 &     0 &     0 & 0 \\
	4 & Gantung           & Gantung       &         0 &    0 & 0 &     0 &     0 & 0 &     0 &     0 & 0 \\
	5 & Simpang Renggiang & Renggiang     &         0 &    0 & 0 &     0 &     0 & 0 &     0 &     0 & 0 \\
	6 & Simpang Pesak     & Simpang Pesak &         0 &    0 & 0 &     1 &     0 & 1 &     1 &     0 & 1 \\
	7 & Dendang           & Dendang       &         0 &    0 & 0 &     2 &     0 & 2 &     2 &     0 & 2 \\
    \midrule
    \multicolumn{3}{l}{JUMLAH KAB.}       &         0 &    0 & 0 &     4 &     1 & 5 &     4 &     1 & 5 \\
    \multicolumn{3}{l}{PROPORSI JENIS KELAMIN} & NULL & NULL &   & 80,00 & 20,00 &   & 80,00 & 20,00 &   \\
    \midrule
    \multicolumn{6}{Y{6cm}}{ANGKA PENEMUAN KASUS BARU (NCDR/ \emph{ NEW CASE DETECTION RATE}) PER 100.000 PENDUDUK} & & & & 6,04 &  1,59 & 3,87 \\
    \bottomrule
\end{tabular}%

}
\vfill
Sumber: Subkoordinator Pengendalian Penyakit Menular\par
