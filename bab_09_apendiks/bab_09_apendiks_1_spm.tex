\chapter{Standar Pelayanan Minimal}
\begin{center}
\renewcommand*{\arraystretch}{1.3}
%\newcolumntype{R}[1]{>{\raggedleft\arraybackslash}p{#1}}
\begin{longtable}{rY{20em}Z{6em}Z{6em}Z{4em}}
% prevent longtable show up twice in TOC. set caption for firsthead, then set empty starred
% caption for next heads. in longtable \toprule should be prepended by \\
\caption{Capaian Standar Pelayanan Minimal (SPM) Bidang Kesehatan Kab. Belitung Timur Tahun \tP}
\\ \toprule
No & JENIS PELAYANAN{*} & PEMBILANG & PENYEBUT{**} & \% \\
\midrule
\endfirsthead
\caption*{}
\\ \toprule
No & JENIS PELAYANAN{*} & PEMBILANG & PENYEBUT{**} & \% \\
\midrule
\endhead

 1 & Pelayanan kesehatan ibu hamil                          &  1.880 &  2.316 &  81,17 \\
 2 & Pelayanan kesehatan ibu bersalin                       &  1.784 &  2.211 &  80,69 \\
 3 & Pelayanan kesehatan bayi baru lahir                    &  1.781 &  2.106 &  84,57 \\
 4 & Pelayanan kesehatan balita                             &  6.532 &  7.077 &  92,30 \\
 5 & Pelayanan kesehatan pada usia pendidikan dasar         & 19.559 & 19.394 & 100,85 \\
 6 & Pelayanan kesehatan pada usia produktif                & 53.786 & 84.553 &  63,61 \\
 7 & Pelayanan kesehatan pada usia lanjut                   &  9.499 & 12.776 &  74,35 \\
 8 & Pelayanan kesehatan penderita hipertensi               & 24.348 & 28.722 &  84,77 \\
 9 & Pelayanan kesehatan penderita diabetes melitus         &  1.689 &  1.801 &  93,78 \\
10 & Pelayanan kesehatan orang dengan gangguan jiwa berat   &    302 &    302 & 100,00 \\
11 & Pelayanan kesehatan orang terduga tuberkulosis         &  1.905 &  1.527 & 124,75 \\
12 & Pelayanan kesehatan orang dengan risiko terinfeksi HIV &  2.920 &  2.347 & 124,41 \\
\bottomrule
\end{longtable}
\par\end{center}

{*}) \emph{Sesuai Permenkes No. 4 Tahun 2019 tentang Standar Teknis Pemenuhan Mutu Pelayanan Dasar Pada Standar Pelayanan
Minimal Bidang Kesehatan}

{**}) \emph{Berdasarkan estimasi dan tidak selalu menggambarkan jumlah yang sebenarnya di populasi}