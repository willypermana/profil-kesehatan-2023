\chapter{Standar Pelayanan Minimal}
\begin{center}
\renewcommand*{\arraystretch}{1.3}
%\newcolumntype{R}[1]{>{\raggedleft\arraybackslash}p{#1}}
\begin{longtable}{rY{20em}Z{6em}Z{6em}Z{4em}}
% prevent longtable show up twice in TOC. set caption for firsthead, then set empty starred
% caption for next heads. in longtable \toprule should be prepended by \\
\caption{Capaian Standar Pelayanan Minimal (SPM) Bidang Kesehatan Kab. Belitung Timur Tahun \tP}
\\ \toprule
No & JENIS PELAYANAN{*} & PEMBILANG & PENYEBUT{**} & \% \\
\midrule
\endfirsthead
\caption*{}
\\ \toprule
No & JENIS PELAYANAN{*} & PEMBILANG & PENYEBUT{**} & \% \\
\midrule
\endhead

 1 & Pelayanan kesehatan ibu hamil                          &  1.892 &  2.219 &  85,26 \\
 2 & Pelayanan kesehatan ibu bersalin                       &  1.907 &  2.118 &  90,03 \\
 3 & Pelayanan kesehatan bayi baru lahir                    &  1.899 &  2.017 &  94,15 \\
 4 & Pelayanan kesehatan balita                             &  7.237 &  7.367 &  98,24 \\
 5 & Pelayanan kesehatan pada usia pendidikan dasar         & 19.435 & 19.684 &  98,74 \\
 6 & Pelayanan kesehatan pada usia produktif                & 74.911 & 84.563 &  88,59 \\
 7 & Pelayanan kesehatan pada usia lanjut                   & 11.350 & 13.052 &  86,96 \\
 8 & Pelayanan kesehatan penderita hipertensi               & 22.597 & 28.832 &  78,37 \\
 9 & Pelayanan kesehatan penderita diabetes melitus         &  1.723 &  1.807 &  95,35 \\
10 & Pelayanan kesehatan orang dengan gangguan jiwa berat   &    303 &    303 & 100,00 \\
11 & Pelayanan kesehatan orang terduga tuberkulosis         &  1.964 &  3.024 &  64,95 \\
12 & Pelayanan kesehatan orang dengan risiko terinfeksi HIV &  3.033 &  3.233 &  93,81 \\
\midrule
\multicolumn{2}{r}{Indeks SPM{***}}                         &        &        & 89,60 \\
                                                     \multicolumn{5}{r}{TUNTAS MADYA} \\
\bottomrule
\end{longtable}
\par\end{center}


{*}) \emph{Sesuai Peraturan Menteri Kesehatan Nomor 4 Tahun 2019 tentang Standar Teknis Pemenuhan Mutu Pelayanan Dasar Pada Standar Pelayanan
Minimal Bidang Kesehatan}

{**}) \emph{Berdasarkan estimasi dan tidak selalu menggambarkan jumlah yang sebenarnya di populasi}

{***}) \emph{Sesuai Peraturan Menteri Dalam Negeri Nomor 59 Tahun 2021 Tentang Penerapan Standar Pelayanan Minimal}