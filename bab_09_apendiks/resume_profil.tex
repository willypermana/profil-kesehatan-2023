\phantomsection
\addcontentsline{lot}{chapter}{\protect\numberline{}Resume Profil}
% Table generated by Excel2LaTeX from sheet 'Resume'
\begin{small}
\begin{longtable}{rY{12em}rrrrY{5em}Y{5em}}
    \multicolumn{8}{c}{RESUME PROFIL KESEHATAN}\\
    \multicolumn{8}{c}{KABUPATEN BELITUNG TIMUR}\\
    \multicolumn{8}{c}{TAHUN \tP}\\
	\\ \toprule

	\multirow{2}{*}{\textbf{No}} & \multirow{2}{*}{\textbf{INDIKATOR}}                & \multicolumn{5}{c}{\textbf{ANGKA/ NILAI}} & \multirow{2}{5em}{\raggedright\textbf{No. Lampiran}} \\
    \cmidrule{3-7}
	&                                                                                 & \textbf{L} & \textbf{P} & \textbf{L + P} & \textbf{Jumlah} & \textbf{Satuan} & \\ \midrule
	\endfirsthead
	\\ \toprule
	\multirow{2}{*}{\textbf{No}} & \multirow{2}{*}{\textbf{INDIKATOR}}                & \multicolumn{5}{c}{\textbf{ANGKA/ NILAI}} & \multirow{2}{5em}{\raggedright\textbf{No. Lampiran}} \\
    \cmidrule{3-7}
    &                                                                                 & \textbf{L} & \textbf{P} & \textbf{L + P} & \textbf{Jumlah} & \textbf{Satuan} & \\ \midrule
    \endhead
    % make bottom border on when table continue to next page
    \midrule
    \endfoot
    \endlastfoot
	\textbf{I} & \textbf{GAMBARAN UMUM}                                               &        &        &                    &          &                                &          \\
	  1 & Luas Wilayah                                                                &        &        &                    & 2.506,90 & Km\textsuperscript{2}          & Tabel 1  \\
	  2 & Jumlah Desa/ Kelurahan                                                      &        &        &                    &       39 & Desa/Kelurahan                 & Tabel 1  \\
	  3 & Jumlah Penduduk                                                             & 65.971 & 62.650 &            128.621 &          & Jiwa                           & Tabel 2  \\
	  4 & Rata-rata jiwa/rumah tangga                                                 &        &        &                    &     2,96 & Jiwa                           & Tabel 1  \\
	  5 & Kepadatan Penduduk /Km\textsuperscript{2}                                   &        &        &                    &    51,31 & Jiwa/Km\textsuperscript{2}     & Tabel 1  \\
	  6 & Rasio Beban Tanggungan                                                      &        &        &                    &    43,94 & per 100 penduduk produktif     & Tabel 2  \\
	  7 & Rasio Jenis Kelamin                                                         &        &        &                    &   105,30 &                                & Tabel 2  \\
	  8 & Penduduk 15 tahun ke atas melek huruf                                       &  99,52 &  98,31 &              98,93 &          & \%                             & Tabel 3  \\
	  9 & Penduduk 15 tahun yang memiliki ijazah tertinggi                            &        &        &                    &          &                                &          \\
		& a. SMP/ MTs                                                                 &  19,62 &  16,82 &              18,26 &          & \%                             & Tabel 3  \\
		& b. SMA/ MA                                                                  &  27,44 &  25,99 &              26,74 &          & \%                             & Tabel 3  \\
		& c. Sekolah menengah kejuruan                                                &   7,25 &   3,67 &               5,52 &          & \%                             & Tabel 3  \\
		& d. Diploma I/ Diploma II                                                    &   0,48 &   0,58 &               0,53 &          & \%                             & Tabel 3  \\
		& e. Akademi/ Diploma III                                                     &   1,28 &    2,3 &               1,77 &          & \%                             & Tabel 3  \\
		& f. S1/ Diploma IV                                                           &    4,4 &   6,04 &               5,19 &          & \%                             & Tabel 3  \\
		& g. S2/ S3 (Master/ Doktor)                                                  &   0,29 &    0,2 &               0,25 &          & \%                             & Tabel 3  \\
	    &                                                                             &        &        &                    &          &                                &          \\ 
    \textbf{II} & \textbf{SARANA KESEHATAN}                                           &        &        &                    &          &                                &          \\
    \textbf{II.1} & \textbf{Sarana Kesehatan}                                         &        &        &                    &          &                                &          \\
	 10 & Jumlah Rumah Sakit Umum                                                     &        &        &                    &        1 & RS                             & Tabel 4  \\
	 11 & Jumlah Rumah Sakit Khusus                                                   &        &        &                    &        0 & RS                             & Tabel 4  \\
	 12 & Jumlah Puskesmas Rawat Inap                                                 &        &        &                    &        4 & Puskesmas                      & Tabel 4  \\
	 13 & Jumlah Puskesmas non-Rawat Inap                                             &        &        &                    &        3 & Puskesmas                      & Tabel 4  \\
	 14 & Jumlah Puskesmas Keliling                                                   &        &        &                    &        0 & Puskesmas keliling             & Tabel 4  \\
	 15 & Jumlah Puskesmas pembantu                                                   &        &        &                    &       15 & Pustu                          & Tabel 4  \\
	 16 & Jumlah Apotek                                                               &        &        &                    &       24 & Apotek                         & Tabel 4  \\
	 17 & Jumlah Klinik Pratama                                                       &        &        &                    &        7 & Klinik Pratama                 & Tabel 4  \\
	 18 & Jumlah Klinik Utama                                                         &        &        &                    &        1 & Klinik Utama                   & Tabel 4  \\
	 17 & RS dengan kemampuan pelayanan gadar level 1                                 &        &        &                    &      100 & \%                             & Tabel 6  \\
	    &                                                                             &        &        &                    &          &                                &          \\ 
    \textbf{II.2} & \textbf{Akses dan Mutu Pelayanan Kesehatan}                       &        &        &                    &          &                                &          \\
	 18 & Cakupan Kunjungan Rawat Jalan                                               & 157,78 & 204,62 &             180,60 &          & \%                             & Tabel 5  \\
	 19 & Cakupan Kunjungan Rawat Inap                                                &   3,93 &   6,05 &               5,01 &          & \%                             & Tabel 5  \\
	 20 & Angka kematian kasar/ \emph{Gross Death Rate (GDR)} di RS                   &  80,30 &  56,92 &              66,27 &          & per 1.000 pasien keluar        & Tabel 7  \\
	 21 & Angka kematian murni/ \emph{Nett Death Rate (NDR)} di RS                    &  25,68 &  34,21 &              30,80 &          & per 1.000 pasien keluar        & Tabel 7  \\
	 22 & \emph{Bed Occupation Rate} (BOR) di RS                                      &        &        &                    &    47,02 & \%                             & Tabel 8  \\
	 23 & \emph{Bed Turn Over} (BTO) di RS                                            &        &        &                    &    46,18 & Kali                           & Tabel 8  \\
	 24 & \emph{Turn of Interval (TOI)} di RS                                         &        &        &                    &     4,19 & Hari                           & Tabel 8  \\
	 25 & \emph{Average Length of Stay (ALOS)} di RS                                  &        &        &                    &     3,76 & Hari                           & Tabel 8  \\
	 26 & Puskesmas dengan ketersediaan obat vaksin \& essensial                      &        &        &                    &      100 & \%                             & Tabel 9  \\
	 27 & Persentase Ketersediaan Obat Essensial                                      &        &        &                    &      100 & \%                             & Tabel 10 \\
	 28 & Persentase puskesmas dengan ketersediaan vaksin IDL                         &        &        &                    &      100 & \%                             & Tabel 11 \\
	 &                                                                                &        &        &                    &          &                                &          \\
	\textbf{II.3} & \textbf{Upaya Kesehatan Bersumberdaya Masyarakat (UKBM)}          &        &        &                    &          &                                &          \\
	 27 & Jumlah Posyandu                                                             &        &        &                    &      134 & Posyandu                       & Tabel 12 \\
	 28 & Posyandu Aktif                                                              &        &        &                    &    99,25 & \%                             & Tabel 12 \\
	 29 & Rasio posyandu per 100 balita                                               &        &        &                    &     1,46 & per 100 balita                 & Tabel 12 \\
	 30 & Posbindu PTM                                                                &        &        &                    &       62 & Posbindu PTM                   & Tabel 12 \\
	 &                                                                                &        &        &                    &          &                                &          \\
	\textbf{III} & \textbf{SUMBER DAYA MANUSIA KESEHATAN}                             &        &        &                    &          &                                &          \\
	 31 & Jumlah Dokter Spesialis                                                     &     10 &      8 &                 18 &          & Orang                          & Tabel 13 \\
	 32 & Jumlah Dokter Umum                                                          &     28 &     34 &                 62 &          & Orang                          & Tabel 13 \\
	 33 & Rasio Dokter (spesialis+umum)                                               &        &        &              62,20 &          & per 100.000 penduduk           & Tabel 13 \\
	 34 & Jumlah Dokter Gigi + Dokter Gigi Spesialis                                  &      1 &      8 &                  9 &          & Orang                          & Tabel 13 \\
	 35 & Rasio Dokter Gigi (termasuk Dokter Gigi Spesialis)                          &        &        &               7,00 &          & per 100.000 penduduk           & Tabel 13 \\
	 36 & Jumlah Bidan                                                                &        &    179 &                    &          & Orang                          & Tabel 14 \\
	 37 & Rasio Bidan per 100.000 penduduk                                            &        & 139,17 &                    &          & per 100.000 penduduk           & Tabel 14 \\
	 38 & Jumlah Perawat                                                              &    138 &    232 &                370 &          & Orang                          & Tabel 14 \\
	 39 & Rasio Perawat per 100.000 penduduk                                          &        &        &             287,67 &          & per 100.000 penduduk           & Tabel 14 \\
	 40 & Jumlah Tenaga Kesehatan Masyarakat                                          &      5 &     14 &                 19 &          & Orang                          & Tabel 15 \\
	 41 & Jumlah Tenaga Kesehatan Lingkungan                                          &      5 &      8 &                 13 &          & Orang                          & Tabel 15 \\
	 42 & Jumlah Tenaga Gizi                                                          &      3 &     19 &                 22 &          & Orang                          & Tabel 15 \\
	 43 & Jumlah Ahli Teknologi Laboratorium Medik                                    &      4 &     23 &                 27 &          & Orang                          & Tabel 16 \\
	 44 & Jumlah Tenaga Teknik Biomedika Lainnya                                      &      4 &      6 &                 10 &          & Orang                          & Tabel 16 \\
	 45 & Jumlah Tenaga Keterapian Fisik                                              &      0 &      7 &                  7 &          & Orang                          & Tabel 16 \\
	 46 & Jumlah Tenaga Keteknisian Medis                                             &      7 &     27 &                 34 &          & Orang                          & Tabel 16 \\
	 47 & Jumlah Tenaga Teknis Kefarmasian                                            &      7 &     19 &                 26 &          & Orang                          & Tabel 17 \\
	 48 & Jumlah Tenaga Apoteker                                                      &      2 &     13 &                 15 &          & Orang                          & Tabel 17 \\
	 49 & Jumlah Tenaga Kefarmasian                                                   &      9 &     32 &                 41 &          & Orang                          & Tabel 17 \\
	    &                                                                             &        &        &                    &          &                                &          \\
	\textbf{IV} & \textbf{PEMBIAYAAN KESEHATAN}                                       &        &        &                    &          &                                &          \\
	 50 & Peserta Jaminan Pemeliharaan Kesehatan                                      &        &        &                    &     0,98 & \%                             & Tabel 19 \\
	 51 & Total anggaran kesehatan                                                    &        & \multicolumn{3}{r}{312.608.364.718,00} & Rp                             & Tabel 20 \\
	 52 & APBD kesehatan terhadap APBD Kabupaten                                      &        &        &                    &    33,94 & \%                             & Tabel 20 \\
	 53 & Anggaran kesehatan per kapita                                               &        &      \multicolumn{3}{r}{2.294.066,00 } & Rp                             & Tabel 20 \\
	    &                                                                             &        &        &                    &          &                                &          \\
	\textbf{V} & \textbf{KESEHATAN KELUARGA}                                          &        &        &                    &          &                                &          \\
	\textbf{V.1} & \textbf{Kesehatan Ibu}                                             &        &        &                    &          &                                &          \\
	 54 & Jumlah Lahir Hidup                                                          &    892 &    896 &              1.788 &          & Orang                          & Tabel 21 \\
	 55 & Angka Lahir Mati (dilaporkan)                                               &   6,68 &   9,94 &               8,32 &          & per 1.000 Kelahiran Hidup      & Tabel 21 \\
	 56 & Jumlah Kematian Ibu                                                         &        &      3 &                    &          & Ibu                            & Tabel 22 \\
	 57 & Angka Kematian Ibu (dilaporkan)                                             &        & 167,79 &                    &          & per 100.000 Kelahiran Hidup    & Tabel 22 \\
	 58 & Kunjungan Ibu Hamil (K1)                                                    &        &  86,27 &                    &          & \%                             & Tabel 24 \\
	 59 & Kunjungan Ibu Hamil (K4)                                                    &        &  81,52 &                    &          & \%                             & Tabel 24 \\
	 60 & Kunjungan Ibu Hamil (K6)                                                    &        &  79,53 &                    &          & \%                             & Tabel 24 \\
	 61 & Persalinan di Fasyankes                                                     &        &  80,69 &                    &          & \%                             & Tabel 24 \\
	 62 & Pelayanan Ibu Nifas KF Lengkap                                              &        &  81,05 &                    &          & \%                             & Tabel 24 \\
	 63 & Ibu Nifas Mendapat Vitamin A                                                &        &  81,14 &                    &          & \%                             & Tabel 24 \\
	 64 & Ibu hamil dengan imunisasi Td2+                                             &        &  86,87 &                    &          & \%                             & Tabel 25 \\
	 65 & Ibu Hamil Mendapat Tablet Tambah Darah 90                                   &        &  81,95 &                    &          & \%                             & Tabel 28 \\
	 66 & Ibu Hamil Mengonsumsi Tablet Tambah Darah 90                                &        &  81,95 &                    &          & \%                             & Tabel 28 \\
	 67 & Bumil dengan Komplikasi Kebidanan yang Ditangani                            &        &  82,25 &                    &          & \%                             & Tabel 32 \\
	 68 & Peserta KB Aktif Modern                                                     &        &        &              78,76 &          & \%                             & Tabel 29 \\
	 69 & Peserta KB Pasca Persalinan                                                 &        &        &              74,40 &          & \%                             & Tabel 31 \\
	 &                                                                                &        &        &                    &          &                                &          \\
	\textbf{V.2} & \textbf{Kesehatan Anak}                                            &        &        &                    &          &                                &          \\
	 70 & Jumlah Kematian Neonatal                                                    &      7 &      8 &                 15 &          & neonatal                       & Tabel 34 \\
	 71 & Angka Kematian Neonatal (dilaporkan)                                        &   7,85 &   8,93 &               8,39 &          & per 1.000 Kelahiran Hidup      & Tabel 34 \\
	 72 & Jumlah Bayi Mati                                                            &     14 &      8 &                 22 &          & bayi                           & Tabel 34 \\
	 73 & Angka Kematian Bayi (dilaporkan)                                            &  15,70 &   8,93 &              12,30 &          & per 1.000 Kelahiran Hidup      & Tabel 34 \\
	 74 & Jumlah Balita Mati                                                          &     16 &     10 &                 26 &          & Balita                         & Tabel 34 \\
	 75 & Angka Kematian Balita (dilaporkan)                                          &  17,94 &  11,16 &              14,54 &          & per 1.000 Kelahiran Hidup      & Tabel 34 \\
	 76 & Bayi baru lahir ditimbang                                                   & 100,00 & 100,00 &             100,00 &          & \%                             & Tabel 37 \\
	 77 & Berat Badan Bayi Lahir Rendah (BBLR)                                        &   5,32 &   5,71 &               5,51 &          & \%                             & Tabel 37 \\
	 78 & Kunjungan Neonatus 1 (KN 1)                                                 &  78,53 &  88,57 &              83,29 &          & \%                             & Tabel 38 \\
	 79 & Kunjungan Neonatus 3 kali (KN Lengkap)                                      &  77,82 &  89,06 &              83,15 &          & \%                             & Tabel 38 \\
	 80 & Bayi yang diberi ASI Eksklusif                                              &        &        &              61,23 &          & \%                             & Tabel 39 \\
	 81 & Pelayanan kesehatan bayi                                                    &  99,43 & 107,03 &             103,03 &          & \%                             & Tabel 40 \\
	 82 & Desa/ Kelurahan UCI                                                         &        &        &                    &      100 & \%                             & Tabel 41 \\
	 83 & Cakupan Imunisasi Campak/ Rubela pada Bayi                                  &  98,41 & 103,14 &             100,65 &          & \%                             & Tabel 43 \\
	 84 & Imunisasi dasar lengkap pada bayi                                           &  98,41 & 103,14 &             100,65 &          & \%                             & Tabel 43 \\
	 85 & Bayi Mendapat Vitamin A                                                     &        &        &              99,42 &          & \%                             & Tabel 45 \\
	 86 & Anak Balita Mendapat Vitamin A                                              &        &        &              95,97 &          & \%                             & Tabel 45 \\
	 87 & Balita Mendapatkan Vitamin A                                                &        &        &              99,42 &          & \%                             & Tabel 45 \\
	 88 & Balita Memiliki Buku KIA                                                    &        &        &              92,67 &          & \%                             & Tabel 46 \\
	 89 & Balita Dipantau Pertumbuhan dan Perkembangan                                &        &        &              71,26 &          & \%                             & Tabel 46 \\
	 90 & Balita ditimbang (D/S)                                                      &  77,66 &  78,20 &              77,92 &          & \%                             & Tabel 47 \\
	 91 & Balita Berat Badan Kurang (BB/U)                                            &        &        &               6,00 &          & \%                             & Tabel 48 \\
	 92 & Balita pendek (TB/U)                                                        &        &        &               4,50 &          & \%                             & Tabel 48 \\
	 93 & Balita Gizi Kurang (BB/TB)                                                  &        &        &               2,28 &          & \%                             & Tabel 48 \\
	 94 & Balita Gizi Buruk (BB/TB)                                                   &        &        &               0,07 &          & \%                             & Tabel 48 \\
	 95 & Cakupan Penjaringan Kesehatan Siswa Kelas 1 SD/ MI                          &        &        &             100,00 &          & \%                             & Tabel 49 \\
	 96 & Cakupan Penjaringan Kesehatan Siswa Kelas 7 SMP/ MTs                        &        &        &             100,00 &          & \%                             & Tabel 49 \\
	 97 & Cakupan Penjaringan Kesehatan Siswa Kelas 10 SMA/ MA                        &        &        &             100,00 &          & \%                             & Tabel 49 \\
	 98 & Pelayanan kesehatan pada usia pendidikan dasar                              &        &        &              99,96 &          & \%                             & Tabel 49 \\
	 &                                                                                &        &        &                    &          &                                &          \\
	\textbf{V.3} & \textbf{Kesehatan Usia Produktif dan Usia Lanjut}                  &        &        &                    &          &                                &          \\
	 99 & Pelayanan Kesehatan Usia Produktif                                          &  53,95 &  92,60 &              72,56 &          & \%                             & Tabel 52 \\
	100 & Catin Mendapatkan Layanan Kesehatan                                         & 109,27 & 109,27 &             109,27 &          & \%                             & Tabel 53 \\
	101 & Pelayanan Kesehatan Usila (60+ tahun)                                       &  77,87 &  94,23 &              80,83 &          & \%                             & Tabel 54 \\
	&                                                                                 &        &        &                    &          &                                &          \\
	\textbf{VI} & \textbf{PENGENDALIAN PENYAKIT}                                      &        &        &                    &          &                                &          \\
	\textbf{VI.1} & \textbf{Pengendalian Penyakit Menular Langsung}                   &        &        &                    &          &                                &          \\
	102 & Persentase orang terduga TBC mendapatkan pelayanan kesehatan sesuai standar &        &        &             103,73 &          & \%                             & Tabel 56 \\
	103 & \emph{Treatment Coverage} TBC                                               &        &        &              22,46 &          & \%                             & Tabel 56 \\
	104 & Cakupan penemuan kasus TBC anak                                             &        &        &              14,49 &          & \%                             & Tabel 56 \\
	105 & Angka kesembuhan BTA+                                                       &  62,96 &  45,95 &              54,84 &          & \%                             & Tabel 57 \\
	106 & Angka pengobatan lengkap semua kasus TBC                                    &  37,04 &  54,05 &             100,00 &          & \%                             & Tabel 57 \\
	107 & Angka keberhasilan pengobatan (\emph{Success Rate}) semua kasus TBC         & 100,00 & 100,00 &             100,00 &          & \%                             & Tabel 57 \\
	108 & Jumlah kematian selama pengobatan tuberkulosis                              &        &        &               3,87 &          & \%                             & Tabel 57 \\
	109 & Penemuan penderita pneumonia pada balita                                    &        &        &              14,40 &          & \%                             & Tabel 58 \\
	110 & Puskesmas yang melakukan tatalaksana standar pneumonia min 60\%             &        &        &                    &   100,00 & \%                             & Tabel 58 \\
	111 & Jumlah Kasus HIV                                                            &      6 &      6 &                 12 &          & Kasus                          & Tabel 59 \\
	112 & Persentase ODHIV Baru Mendapat Pengobatan ARV                               &        &        &              85,71 &          & \%                             & Tabel 60 \\
	113 & Persentase Penderita Diare pada Semua Umur Dilayani                         &        &        &              24,02 &          & \%                             & Tabel 61 \\
	114 & Persentase Penderita Diare pada Balita Dilayani                             &        &        &              24,02 &          & \%                             & Tabel 61 \\
	115 & Persentase Ibu hamil diperiksa Hepatitis                                    &        &        &              86,27 &          & \%                             & Tabel 62 \\
	116 & Persentase Ibu hamil diperiksa Reaktif Hepatitis                            &        &        &               1,80 &          & \%                             & Tabel 62 \\
	117 & Persentase Bayi dari Bumil Reakif Hepatitis Diperiksa                       &        &        &             100,00 &          & \%                             & Tabel 63 \\
	118 & Jumlah Kasus Baru Kusta (PB+MB)                                             &      2 &      1 &                  3 &          & Kasus                          & Tabel 64 \\
	119 & Angka penemuan kasus baru kusta (NCDR)                                      &   3,03 &   1,60 &               2,33 &          & per 100.000 penduduk           & Tabel 64 \\
	120 & Persentase Kasus Baru Kusta anak < 15 Tahun                                 &        &        &              33,33 &          & \%                             & Tabel 65 \\
	121 & Persentase Cacat Tingkat 0 Penderita Kusta                                  &        &        &              66,67 &          & \%                             & Tabel 65 \\
	122 & Persentase Cacat Tingkat 2 Penderita Kusta                                  &        &        &                  0 &          & \%                             & Tabel 65 \\
	123 & Angka Cacat Tingkat 2 Penderita Kusta                                       &        &        &                  0 &          & per 100.000 penduduk           & Tabel 65 \\
	124 & Angka Prevalensi Kusta                                                      &        &        &               0,39 &          & per 10.000 Penduduk            & Tabel 66 \\
	125 & Penderita Kusta PB Selesai Berobat (RFT PB)                                 &        &        &                NUL &          & \%                             & Tabel 67 \\
	126 & Penderita Kusta MB Selesai Berobat (RFT MB)                                 &        &        &              66,67 &          & \%                             & Tabel 67 \\
	&                                                                                 &        &        &                    &          &                                &          \\
	\textbf{VI.2} & \textbf{Pengendalian Penyakit Yang Dapat Dicegah dengan Imunisasi}&        &        &                    &          &                                &          \\
	127 & AFP Rate (non polio) < 15 tahun                                             &        &        &                  0 &          & per 100.000 penduduk <15 tahun & Tabel 68 \\
	128 & Jumlah kasus difteri                                                        &      0 &      0 &                  0 &          & Kasus                          & Tabel 69 \\
	129 & \emph{Case Fatality Rate} difteri                                           &        &        &               0,00 &          & \%                             & Tabel 69 \\
	130 & Jumlah kasus pertusis                                                       &      0 &      0 &                  0 &          & Kasus                          & Tabel 69 \\
	131 & Jumlah kasus tetanus neonatorum                                             &      0 &      0 &                  0 &          & Kasus                          & Tabel 69 \\
	132 & \emph{Case Fatality Rate} tetanus neonatorum                                &        &        &               0,00 &          & \%                             & Tabel 69 \\
	133 & Jumlah kasus hepatitis B                                                    &      0 &      0 &                  0 &          & Kasus                          & Tabel 69 \\
	134 & Jumlah kasus suspek campak                                                  &      0 &      1 &                  1 &          & Kasus                          & Tabel 69 \\
	135 & \emph{Incidence Rate} suspek campak                                         &   0,00 &   0,78 &               0,78 &          & per 100.000 penduduk           & Tabel 69 \\
	136 & KLB ditangani < 24 jam                                                      &        &        &                    &      NUL & \%                             & Tabel 70 \\
	&                                                                                 &        &        &                    &          &                                &          \\
	\textbf{VI.3} & \textbf{Pengendalian Penyakit Tular Vektor dan Zoonotik}          &        &        &                    &          &                                &          \\
	137 & Angka kesakitan (\emph{Incidence Rate})DBD                                  &        &        &              21,77 &          & per 100.000 penduduk           & Tabel 72 \\
	138 & Angka kematian (\emph{Case Fatality Rate}) DBD                              &   0,00 &   0,00 &               0,00 &          & \%                             & Tabel 72 \\
	139 & Angka kesakitan malaria (\emph{Annual Parasite Incidence)}                  &        &        &                  0 &          & per 1.000 penduduk             & Tabel 73 \\
	140 & Konfirmasi laboratorium pada suspek malaria                                 &        &        &             100,00 &          & \%                             & Tabel 73 \\
	141 & Pengobatan standar kasus malaria positif                                    &        &        &                NUL &          & \%                             & Tabel 73 \\
	142 & \emph{Case Fatality Rate} malaria                                           &    NUL &    NUL &                NUL &          & \%                             & Tabel 73 \\
	143 & Penderita kronis filariasis                                                 &     13 &      1 &                 14 &          & Kasus                          & Tabel 74 \\
	144 & Jumlah Kasus Covid-19                                                       &        &        &                614 &          & Kasus                          & Tabel 84 \\
	145 & CFR (\emph{Case Fatality Rate}) Covid-19                                    &        &        &               3,58 &          & \%                             & Tabel 84 \\
	146 & Cakupan Total Vaksinasi Covid-19 Dosis 1                                    &        &        &              10,92 &          & \%                             & Tabel 86 \\
	147 & Cakupan Total Vaksinasi Covid-19 Dosis 2                                    &        &        &               6,26 &          & \%                             & Tabel 87 \\
	&                                                                                 &        &        &                    &          &                                &          \\
	\textbf{VI.4} & \textbf{Pengendalian Penyakit Tidak Menular}                      &        &        &                    &          &                                &          \\
	148 & Penderita Hipertensi Mendapat Pelayanan Kesehatan                           &  63,40 & 107,18 &              84,77 &          & \%                             & Tabel 75 \\
	149 & Penyandang DM  mendapatkan pelayanan kesehatan sesuai standar               &        &        &              93,80 &          & \%                             & Tabel 76 \\
	150 & Pemeriksaan IVA pada perempuan usia 30-50 tahun                             &        &  19,49 &                    &          & \% perempuan usia 30-50 tahun  & Tabel 77 \\
	151 & Persentase IVA positif pada perempuan usia 30-50 tahun                      &        &   0,05 &                    &          & \%                             & Tabel 77 \\
	152 & Pemeriksaan payudara (SADANIS) pada perempuan 30-50 tahun                   &        &  19,61 &                    &          & \%                             & Tabel 77 \\
	153 & Persentase tumor/benjolan payudara pada perempuan 30-50 tahun               &        &   0,05 &                    &          & \%                             & Tabel 77 \\
	154 & Pelayanan Kesehatan Orang dengan Gangguan Jiwa Berat                        &        &        &             100,00 &          & \%                             & Tabel 78 \\
	&                                                                                 &        &        &                    &          &                                &          \\ 
	 \textbf{VII} & \textbf{KESEHATAN LINGKUNGAN}                                     &        &        &                    &          &                                &          \\
	155 & Sarana Air Minum yang DiawasiI/ Diperiksa Kualitas Air Minumnya Sesuai Standar (Aman)&   &    &                    &    46,88 & \%                             & Tabel 79 \\
	156 & KK Stop BABS (SBS)                                                          &        &        &                    &   100,00 & \%                             & Tabel 80 \\
	157 & KK dengan Akses terhadap Fasilitas Sanitasi yang Layak                      &        &        &                    &    97,06 & \%                             & Tabel 80 \\
	158 & KK dengan Akses terhadap Fasilitas Sanitasi yang Aman                       &        &        &                    &     3,45 & \%                             & Tabel 80 \\
	159 & Desa/ Kelurahan Stop BABS (SBS)                                             &        &        &                    &   100,00 & \%                             & Tabel 81 \\
	160 & KK Cuci Tangan Pakai Sabun (CTPS)                                           &        &        &                    &    59,78 & \%                             & Tabel 81 \\
	161 & KK Pengelolaan Air Minum dan Makanan Rumah Tangga (PAMMRT)                  &        &        &                    &    66,32 & \%                             & Tabel 81 \\
	162 & KK Pengelolaan Sampah Rumah Tangga (PSRT)                                   &        &        &                    &    18,10 & \%                             & Tabel 81 \\
	163 & KK Pengelolaan Limbah Cair Rumah Tangga (PLCRT)                             &        &        &                    &     2,91 & \%                             & Tabel 81 \\
	164 & Desa/ Kelurahan 5 Pilar STBM                                                &        &        &                    &     0,00 & \%                             & Tabel 81 \\
	165 & KK Pengelolaan Kualitas Udara dalam Rumah Tangga (PKURT)                    &        &        &                    &    60,90 & \%                             & Tabel 81 \\
	166 & KK Akses Rumah Sehat                                                        &        &        &                    &     2,88 & \%                             & Tabel 81 \\
	167 & Tempat Fasilitas Umum (TFU) yang Dilakukan Pengawasan Sesuai Standar        &        &        &                    &    85,21 & \%                             & Tabel 82 \\
	168 & Tempat Pengelolaan Pangan (TPP) Jasa Boga yang Memenuhi Syarat Kesehatan    &        &        &                    &    63,64 & \%                             & Tabel 83 \\ 
	 \bottomrule
\end{longtable}%
\end{small}