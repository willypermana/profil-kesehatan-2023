\phantomsection
% add \protect\numberline{} to get nice indentation in the list of tables
\addcontentsline{lot}{section}{\protect\numberline{}Tabel 1 - Luas wilayah, jumlah desa/ kelurahan, jumlah penduduk, jumlah rumah tangga dan kepadatan penduduk}
\label{tabel-01}
\ra{1.3}
% Table generated by Excel2LaTeX from sheet '1'
{\centering
\begin{tabular}{clrrrrrrrr}
    \multicolumn{10}{l}{Tabel 1}\\
    \multicolumn{10}{c}{LUAS WILAYAH, JUMLAH DESA/ KELURAHAN, JUMLAH PENDUDUK, JUMLAH RUMAH TANGGA,}\\
    \multicolumn{10}{c}{DAN KEPADATAN PENDUDUK MENURUT KECAMATAN.}\\
    \multicolumn{10}{c}{\namaKabupatenKapital}\\
    \multicolumn{10}{c}{TAHUN \tP}\\
    \toprule
    \hrulefill
    \multirow{2}[0]{*}{NO} & \multirow{2}[0]{*}{KECAMATAN} & \multirow{2}[0]{*}{\parbox{6em}{\raggedleft LUAS WILAYAH (Km\textsuperscript{2})}} & \multicolumn{3}{X{16em}}{JUMLAH} & \multirow{2}[0]{*}{\parbox{6em}{\raggedleft JUMLAH PENDUDUK}} & \multirow{2}[0]{*}{\parbox{6em}{\raggedleft JUMLAH RUMAH TANGGA }} & \multirow{2}[0]{*}{\parbox[r]{6em}{\raggedleft RATA-RATA JIWA/ RUMAH TANGGA }} & \multirow{2}[0]{*}{\parbox{6em}{\raggedleft KEPADATAN PENDUDUK PER Km\textsuperscript{2}}} \\
%    \cmidrule(l{2pt}r{2pt}){4-4}\cmidrule(l{2pt}r{2pt}){5-5}\cmidrule(l{2pt}r{2pt}){6-6}
	\cmidrule{4-6}
    & & & DESA & KELURAHAN & \multicolumn{1}{Z{6em}}{DESA + KELURAHAN} & & & & \\
    \midrule
    \emph{1} & \emph{2} & \emph{3} & \emph{4} & \emph{5} & \emph{6} & \emph{7} & \emph{8} & \emph{9} & \emph{10}\\
    \midrule
	1 & Manggar           & 229,0 & 9 & 0 & 9 & 40.007 & 13.459 & 2,97 & 174,70 \\
    2 & Damar             & 236,9 & 5 & 0 & 5 & 13.336 &  4.551 & 2,93 &  56,29 \\
    3 & Kelapa Kampit     & 498,5 & 6 & 0 & 6 & 19.306 &  6.633 & 2,91 &  38,73 \\
    4 & Gantung           & 546,3 & 7 & 0 & 7 & 29.010 &  9.641 & 3,01 &  53,10 \\
    5 & Simpang Renggiang & 390,7 & 4 & 0 & 4 &  7.586 &  2.721 & 2,79 &  19,42 \\
    6 & Simpang Pesak     & 362,2 & 4 & 0 & 4 &  8.619 &  2.883 & 2,99 &  23,80 \\
    7 & Dendang           & 243,3 & 4 & 0 & 4 & 10.757 &  3.558 & 3,02 &  44,21 \\
    \midrule
    \multicolumn{2}{l}{JUMLAH KAB.}& 2.506,9 & 39 & 0 & 39 & 128.621 & 43.446 & 2,96 & 51,31 \\
    \bottomrule
\end{tabular}%

}

\vfill
Sumber: \\
- Dinas Kependudukan dan Pencatatan Sipil \namaKabupaten \\
- Proyeksi internal berdasar data Dinas Kependudukan dan Pencatatan Sipil Kabupaten Belitung Timur tahun 2021 \par